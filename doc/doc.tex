\documentclass[a4paper,oneside,12pt]{article}
\usepackage{polski}
\usepackage[T1]{fontenc}
\usepackage[utf8]{inputenc}
\usepackage{geometry}
\usepackage{amssymb}
\usepackage{amsmath}
\usepackage{textgreek}

\usepackage[printwatermark]{xwatermark}
\usepackage{xcolor}
\usepackage{graphicx}
\usepackage{lipsum}
\usepackage[ampersand]{easylist}

\newwatermark[allpages,color=gray!30,angle=45,scale=3,xpos=0,ypos=0]{SZKIC}

\title{Projekt WEDT, wersja robocza}
\author{
	Michał Krawczak \\
	Adrian Więch \\
	Jakub Zawiślak
}
\date{\today}

\begin{document}

\maketitle
\newpage
\tableofcontents
\newpage

\setlength\parindent{0pt}
\setlength{\parskip}{0.25em}

\section{Temat i zakres projektu}

W ramach projektu badaliśmy możliwość wykorzystania technik uczenia maszynowego do ekstrakcji danych osób i instytucji z faktur.

\section{Opis zbioru testowego}

Zbiór testowy stanowią pliki tekstowe będące rezultatem przetworzenia zeskanowanych faktur przez program typu OCR.

\section{Badane algorytmy}

W ramach projektu zbadaliśmy dwa algorytmy:

\begin{easylist}[itemize]
& Prosty algorytm oparty na N-gramach
& Algorytm oparty na technice losowych lasów decyzyjnych (RDF)
\end{easylist}

Ponadto stworzyliśmy program o nazwie Tokenizer, służący do dzielenia faktur na tokeny i oznaczania charakterystycznych dla faktur typów tokenów. Jest on używany do wstępnego przetworzenia wejścia dla obu algorytmów.

\section{Opis programu Tokenizer}

\subsection{Podział na tokeny}

Pierwszym etapem przetwarzania faktury jest podział na tokeny. Tokeny powinny być, w miarę możliwości, odrębnymi jednostkami semantycznymi. 

Najprostszym rozwiązaniem mógłby być użycie spacji jako separatorów. Takie podejście ma jednak wady:

\begin{easylist}[itemize]
& W plikach tekstowych wygenerowanych przez OCR często brakuje spacji - szczególnie bezpośrednio przed i po znakach interpunkcyjnych, takich jak kropka, przecinek, nawias czy dwukropek. 
& Często chcielibyśmy traktować jako jeden token coś, co zawiera w sobie spacje. Przykładami mogą być wielowyrazowe nazwy miejscowości (np. Mińsk Mazowiecki) czy też numery kont bankowych podzielone na 4-cyfrowe bloki odseparowane spacjami (np. 12 3456 7890 0000 0000 0000 0001). Nie wszystkie spacje powinny być zatem uwzględniane jako separatory.
\end{easylist}

W przypadku pierwszym rozwiązaniem jest używanie jako separatorów nie tylko spacji, lecz także: dwukropków, kropek, nawiasów i przecinków. To jednak powoduje kolejny problem: podobnie jak spacje, te znaki w określonych sytuacjach nie powinny być separatorami. Na przykład, jeśli mamy dwa ciągi cyfr oddzielone kropką (123.45), to prawdopodobnie mamy do czynienia z liczbą wrac z rozwinięciem dziesiętnym. Podobnie, trzy grupy po dwie cyfry (01.01.16) prawdopodobnie stanowią datę.

Powyższe obserwacje doprowadziły do następującego algorytmu:
\begin{easylist}[itemize]
& Dla każdego znaku w dokumencie należącego do zbioru: {spacja, przecinek, kropka, nawias, średnik, nawias zamykający, nawias otwierający}:
&& Na podstawie otoczenia znaku stwierdź, czy wchodzi on w skład jednego z następujących tokenów:
&&& Nazwa miejscowości
&&& Nazwa ulicy
&&& Inne zesłownikowane wartości, jak np. "S. A." czy "Sp. z o.o."
&&& Numer konta
&&& Data
&&& Liczba dziesiętna
&&& NIP
&& Jeśli nie, to podziel tekst w tym miejscu
& Z wyniku powyższej operacji usuń puste tokeny
\end{easylist}

\subsection{Oznaczanie tokenów}

Każdy z tokenów jest przepuszczany przez szereg testów, ustawionych według priorytetu, z których każdemu testowi odpowiada inny typ tokenu. Jeśli którykolwiek test się powiedzie, token zostaje przyporządkowany do odpowiedniego typu. Testy są podzielone na dwa rodzaje:
\begin{easylist}[itemize]
& Słowniki
& Intuicje
\end{easylist} 
 
Test słownikowy polega na przejrzeniu danego słownika i sprawdzeniu, czy dany token się w nim znajduje. Do projektu użyliśmy następujących publicznie dostępnych słowników:
\begin{easylist}[itemize]
& Słownik nazw ulic
& Słownik nazw miejscowości
& Słownik nazw jednostek samorządu terytorialnego (gminy, powiaty etc.)
& Słownik imion
& Słownik nazwisk
\end{easylist}

\section{Podejście oparte na N-gramach}


\section{Podejście oparte na losowym lesie decyzyjnym}

\end{document}
